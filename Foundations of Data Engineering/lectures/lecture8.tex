\section{Graph databases}
A graph $G = (V, E)$ is a set of edges and vertices which can have different properties. Edges can also have intermediate ramifications between nodes.

Useful information regarding a graph can be:
\begin{itemize}
	\item Degree;
	\item Centrality;
	\item Betweeness;
	\item todo
\end{itemize}

Graph databases can be implemented using semantic data in triples, introducing properties of nodes.

There is no defined standard for structure of graphs, in fact they can also be implemented using relational databases, yet some tools are more popular then others. 

Queries on graph databases do not strictly return a set of tuples: they allow pattern matching, algorithms, aggregation, or properties of graph and its elements.

\subsection{Cypher}
Cypher is one of the most supported query languages for graphs. Its syntax allows nodes and edges to have a type along with some attributes.

Neo4j, implementing Cypher, is a graph database offering dynamic visualization and machine learning algorithms, but it kind of sucks.

\subsection{Pregel/Giraph}
Pregel is Google's iterative graph processing system for distributed data, while Giraph is its open source alternative.

They work using Gremlin, a functional graph traversal language, which with one operation lets a vertex modify its state and propagate information.

The reason why this tool has been developed is because MapReduce systems do not allow operations on a vertex level. In addition, MapReduce requires transferring state to all nodes.

\subsection{Turi}
todo

\subsection{Graph algorithms}
BFS is one of the most important graph traversal algorithms to search a node 

Dijkstra yeah we know that cmon Neumann do databases

% pairing heap


