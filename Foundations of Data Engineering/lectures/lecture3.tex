\section{SQL}
SQL is a declarative language (orders to execute a query without specifying how) to interact with databases and retrieve data. Optimization is not only on the query level, but also on storage and data types: for instance, strings can be managed adding a pointer to their size (if they are long). 

Various SQL dialects have slight differences between semantics, but there are general standards starting from older versions. Different databases might also have different type management, operations and speed.

Another common problem with databases is handling null values while evaluating boolean expressions: NULL and FALSE results in FALSE, therefore this may give an inaccurate representation of information. On the other hand, NOT NULL columns are beneficial for optimization (NULL columns have a reserved memory space for this possibility).

Some PostgreSQL data types:
\begin{itemize}
	\item Numeric, slow arbitrary precision numbers with unbounded size;
	\item Float, mobile floating point (4 bytes) or double precision (8 bytes);
	\item Strings of variable length (pre-allocated or dynamic);
	\item Other common types such as \texttt{bytea}, \texttt{timestamp} and \texttt{interval}.
\end{itemize} 

\subsection{SQL commands}
Standard SQL commands are for instance SELECT, FROM, WHERE, GROUP BY, ORDER BY. Those can be found in normal queries or subqueries, which allow to temporarily subset data without saving it to memory.

The \texttt{\textbackslash copy} statement allows to import data from a text file, but can be slow with large datasets since it requires a full scan.

Regular expression matching is performed through LIKE or $\sim$, but cannot run in linear time since the construction of the NFA is exponential in the size of input. 

Random samples can be extracted with various methods, of which the most accurate in randomness (Bernoulli) is also the slowest.

UNION, EXCEPT and INTERSECT perform set operations, with the keyword ALL to keep duplicates.

Views create faster reusable tables, but are seen globally and this could cause issues with naming. Another way to optimize through a temporary table is \texttt{WITH}, a common table expression.

\subsection{Regular expression matching}


\subsection{Query decorrelation}
Often queries are easier to formulate using subqueries, yet this can take a toll on performance: since SQL is a declarative language, the best evaluation strategy is up to the DBMS.

A subquery is correlated if it refers to tuples from the outer query. The execution of a correlated query is poor, since the inner one gets computed for every tuple of the outer one, causing a quadratic running time.

Example (from TCP-H dataset):
\begin{lstlisting}[language=SQL]
select avg(l_extendedprice)
from lineitem l1
where l_extendedprice =
	(select min(l_extendedprice)
	from lineitem l2
	where l1.l_orderkey = l2.l_orderkey;)
\end{lstlisting}

This query can be rewritten to avoid correlation:
\begin{lstlisting}[language=SQL]
select avg(l_extendedprice)
from lineitem l1,
	(select min(l_extendedprice) m, l_orderkey
	from lineitem
	group by l_orderkey) l2
where l1.l_orderkey = l2.l_orderkey
and l_extendedprice = m;
\end{lstlisting}

The new query is much more efficient, yet not as intuitive. Compilers sometimes manage to automatically decorrelate, but correlations can be complex (inequalities, disjunctions).

A join dependency allows to recreate a table by joining other tables which have subsets of the attributes (relationships are independent of each other). In other words, a table subject to join dependency can be decomposed in other tables and created again joining them.

Dependent joins rely on nested loop evaluations, which are very expensive: the goal of unnesting is to eliminate them all.

The easiest practice consists in pulling predicates up and creating temporary tables inside FROM instead of WHERE, evaluating the subquery for all possible bindings simultaneously. 

The subquery in some cases is a set without duplicates which allows equivalence between dependent join and bindings of free variables, therefore dependency can be removed.

Sometimes attributes of the set can be inferred, computing a superset of the values and eliminating them with the final join. This avoids computing the subquery, but causes a potential loss of pruning the evaluation tree.

The approach is overall always working with relational algebra, yet not necessarily on the SQL level: it can add memory overhead which cannot always be removed, despite the ideal linear computational time.

\subsection{Recursive CTEs}
CTEs are performed only once within each query and destroyed as soon as the query ends on Postgres, yet other database systems may differ in implementation. 

Recursive CTEs traverse hierarchical data of arbitrary length, and consist in a base case (stop condition) and its recursive step. To manipulate values before the previous one, it is useful to store them in columns. It is possible to increment counters within RCTEs, and create trees. 

To avoid infinite loops, using \texttt{UNION} instead of \texttt{UNION ALL} allows to remove duplicates and checking before writing a new value.






\section{Autonomous database systems}
Autonomous database systems are designed to remove the burden of managing DBMS from humans, focusing on physical database design and configuration/query tuning. The first attempts to program self-adaptive systems were in the 1970s, and have now evolved to learned components. 

Indexing has been replacing with a neural network which predicts the location of a key, and transaction are scheduled through the machines by learned scheduling with unsupervised clustering. The most promising innovation is learned planning, deep learning algorithms which help running the query planner, estimating the best execution order of the operations. 

A self driving DBMS is therefore a system that configures, manages and optimizes itself for the target database and its workload, using an objective function (throughput, latency) and some constraints. The two components of this are an agent (the one who makes decisions and learns over time) and an environment, the subject of the actions. Environments have a state used as feedback for the agent. This system can help exploring unknown configurations (and their consequences) and generalizing over applications or hardware.

\subsection{State modeling}
State modeling concerns the representation of the environment state, and has to be performed in an accurate way. The model is stochastic, non stationary and episodic (there is a fixed endpoint), and it can be represented with a Markov decision process model. The entire history of the DBMS is encoded in a state vector: its contents, design, workload and resources.

The table state contains number of tuples, attributes, cardinality and other values which altogether contribute to the choice of specific implementations (for instance indexes), but changes are constrained since the number of dimensions of the feature vectors always has to be the same. The content is approximated through a state vector, therefore there is no precise information regarding how the actual table looks like. 

The knob configuration state depends on the machine, and is not really scalable. Not every configuration is applicable to servers, but one potential solution is to store hardware resources according to their percentages in respect of the available amount. 

Feature vectors can be reduced (PCA), exacerbating instability, and hierarchical models help isolating components to reduce the propagation of changes. 

Acquisition of data is done through targeted experiments while the system is running in production, training the model with end-to-end benchmarks of sample workloads. Instead of running the full system, micro-benchmarks can be run on a subset of components. This method is more effective and requires less iteration to converge. 

To avoid slowing down the entire system, training is performed on replicas (master-slave architecture) having an agent recognize the resources and eventually propagating the best changes to the master. The application server primarily communicates with the master through reads and writes, and obtaining a complete view of all the operation is sometimes hard since not everything is sent to the replicas (failed queries, for instance). 

An alternative is imitation learning: the model observes a tuned DBMS and tries to replicate those decisions. The state of the database still needs to be captured to extrapolate why changes have been applied, and training data will be sparse or noisy.

\subsection{Reward observation}
Rewards are classified as short-term and long-term, the latter more problematic since it is difficult to predict the workload trends. Transient hardware problems are hard to detect and could mask the true reward of an action, so current benchmarks have to be compared with historical data to reconsider recent events. 

Many systems concern both OLTP and OLAP workloads, and changes in the objective functions may have a negative impact on either. Generally multiple policies define preference of one over th other in mixed environments. 

Other common problems regard action selection policies, transparency and human interaction. 


\section{Key-value storage}
One of the benefits of this kind of storage is the lack of schema: they are defined with human readable languages (XML, SVG) for structured and semi-structured data. The possible formats vary from free text to relational models, the latter less common; data can also be compressed or uncompressed. 

For instance, possible choices for configuration files are:
\begin{enumerate}
	\item JSON, although kind of hard to write long files;
	\item YAML, really complicated;
	\item TOML, with minimal syntax therefore easy.
\end{enumerate}

\subsection{XML}
XML documents have a rooted hierarchical structure named Document Object Model, where elements are declared between tags with eventual attributes. Those documents allow to store reasonably large information, queried in a declarative language (XPath, XQuery) which guarantees results appearing in the same order as they were saved. Validation requires constraints which imply the choice between a simple grammar or a more powerful one. Grammar is only defined for tags and attributes, not contents. 

An XPath is the tree containing elements subject of a query: it stores information about preceding nodes, siblings, descendants, parents and ancestors in the DOM. It can be navigated through the syntax \texttt{axis::node[predicate]/@attr} also allowing retrieval of a set related to the matching node (for instance \texttt{:.} indicates parent). 

Grammar definition and DTD are defined in the headers of a document, although parsing can still be difficult due to malfunctioning web pages. 

\subsection{JSON}
JSON is a language similar to XML, but closer to a relational dataase: its structure consists in objects and arrays recursively nested in an arbitrarily complex way. It can be used for a broad range of data types and evaluated like JavaScript (not the best practice for security issues). Keys are between quotes to avoid referencing variables. 

Data can be accessed through indexes and JavaScript syntax.

\subsection{Other}
Schema is similar to XSD and has the same syntax as JSON, yet it is rarely used.

Transact-SQL is the Microsoft JSON query language.

