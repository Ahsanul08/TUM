\section{Tools and techniques for large-scale data processing}
Data processing, especially in the age of big data, can be done with many different approaches and techniques: it is handled with software layers, which have different principles and use cases. All digital data created reached 4 zettabytes in 2013, needing 3 billion drives for storage.

Scientific paradigms concerning collection and analysis have developed as well, including a preprocessing part consisting in observations, modeling and simulations.

Big data is a relative term: each field has its own definition, but it's usually hard to understand. A model might be as complicated as its information, and tools need to be reinvented in a different context as the world evolves.

\subsection{Volume and privacy}
Volume is one of the most important challenges when dealing with big data: it is not often possible to store it on a single machine, therefore it needs to be clustered and scaled. 

Supercomputers (and AWS) are extremely expensive, and have a short life span, making them an unfeasible option. In fact, their usage is oriented towards scientific computing and assumes high quality hardware and maintenance. Programs are written in low-level languages, taking months to develop, and rely extensively on GPUs.

Cluster computing, on the other hand, uses a network of many computers to create a tool oriented towards cheap servers and business applications, although more unreliable. Since it is mostly used to solve large tasks, it relies on parallel database systems, NoSQL and MapReduce to speed up operations. 

Cloud computing is another different instance which uses machines operated by a third party in a data center, renting them by the hour. This often raises controversies about the costs, since machines rarely operate at more than 30\% capacity.

Resources in cloud computing can be scaled as requests dictate, providing elastic provisioning at every level (SaaS, IaaS, PaaS). Higher demand implies more instances, and vice versa.

The downside are the proprietary APIs with lock-in that makes customers vulnerable to price increases, and local laws might prohibit externalizing data processing. Providers might control data in unexpected ways, and have to grant access to the government - privilege which might be overused.

Privacy needs to be guaranteed, stated and kept-to: numerous web accounts get regularly hacked, and there are uncountable examples of private data being leaked. 

Performance is strictly linked to low latency, which works in function of memory, CPU, disk and network. Google, for instance, rewards pages that load quickly. 

\subsection{Velocity}
Velocity is the problematic speed of data: it is generated by an endless stream of events, with no time for heavy indexing leading to developments of new data stream technologies. 

Storage is relatively cheap, but accessing and visualizing is next to impossible. The cause of this are repeated observations, such as locations of mobile phones, motivating stream algorithms. 

\subsection{Variety}
Variety represent big data being inconclusive and incomplete: to fix this problem, simple queries are ineffective and require a machine learning approach (data mining, data cleaning, text analysis). This generates technical complications, while complicating cluster data processing due to the difficulty to partition equally.

Big data typically obeys a power law: modeling the head is easy, but may not be representative of the full population. Most items take a small amount of time to process, but a few require a lot of time; understanding the nature of data is the key. 

Distributed computation is a natural way to tackle Big Data. MapReduce operates over a cluster of machines encouraging sequential, disk-based localized processing of data. The power laws applies, causing uneven allocation of data to nodes (the head would go on one or two workers, making them extremely slowly) and turning parallel algorithms to sequential. 

