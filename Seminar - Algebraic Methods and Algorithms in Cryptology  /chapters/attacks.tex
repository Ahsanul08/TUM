\section{Other attacks}
Attacks against discrete logarithm involve finding the integer $x$ for a given $\alpha$ and $\beta$ in $G$ such that:
$$\beta = \alpha \circ \alpha \circ \dots \circ \ alpha = \alpha^x$$

In 1997, a lower bound for discrete logarithms was given: any generic algorithm that solves with high probability the problem must perform at least $\Omega(p^{1/2})$ group operations, where $p$ is the largest prime dividing the cardinality of the group. The same result holds for non cyclic groups and Diffie-Hellman protocol.

The exact difficulty of computing this is still unknown: despite the existence of algorithms, there could be better or more powerful ones which still are undiscovered (similarly to integer factorization). 

Generic algorithms are methods which only rely on the group operation $\circ$, without needing further algebraic structure. This property makes them applicable in any cyclic group.

\subsection{Brute force search}
Brute-force is the most naive and costly generic algorithm to obtain the discrete logarithm $\log_\alpha\beta$: all powers of the generator $\alpha$ are computed \textbf{successively}, until the result equals $\beta$.

For a random logarithm $x$, on average the correct solution is expected to be found after checking \textit{half of the possible values}. This gives a complexity of $O(|G|)$, linear in the cardinality of the group.

To avoid this kind of attack to be successful, therefore, the cardinality of groups must be sufficiently large, without any other particular measure. 

In the case of $\mathbb{Z}^*_p$ with $p$ prime, approximately $\frac{p-1}{2}$ tests are required to compute the discrete logarithm, meaning that $|G| = p - 1$ should be in the order of at least $2^{80}$ to make a brute force infeasible using the modern hardware.

\subsection{Shank's method}
Shank's method is another generic algorithm which allows to reduce the time of a brute force search, at the trade off of occupying more storage. It is a meet-in-the-middle attack, storing intermediate values by trying to successively crack the multiple steps of discrete logarithm encryption.

The procedure is based on rewriting the discrete logarithm in a two-digit representation:
$$x = x_gm = x_b \qquad 0 \leq x_g, x_b < m$$
The value $m$ is chosen to be approximately the square root of the cardinality of the set, i.e.\ $m = \lceil{\sqrt{|G|}}\rceil$. Discrete logarithm can then be stated as $\beta = \alpha^x = \alpha^{x_gm + x_b}$ which leads to:
$$\beta \cdot (\alpha^{-m})^{x_g} = \alpha^{x_b}$$ 
The core idea of the algorithm is using a divide-and-conquer approach to find $x_a$ and $x_b$ separately, in two phases:
\begin{enumerate}
	\item Baby-step, where all the values $\alpha^{x_b}$ with $0 \leq x_b < m$ are computed and stored with approximately $m \approx \sqrt{|G|}$ operations;
	\item Giant-step, in which the program checks for all $x_g$ in $0 \leq x_g < m$ whether this condition is fulfilled: 
	$\beta \cdot (\alpha^{-m})^g \stackrel{?}{=} \alpha^{x_b}$ for some stored entry $\alpha^{x_b}$ that was computed during the previous step.
\end{enumerate}
In case of a match, i.e.\ $\beta \cdot (\alpha^{-m})^{x_g, 0} = \alpha^{x_b, 0}$ for some pair $(x_{g, 0}, x_{b, 0})$ the discrete logarithm is:
$$x = x_{g, 0}m + x_{b, 0}$$
Further speed up the method can be obtained using efficient lookup schemes, such as \textbf{hash tables} which allow constant time search.

This algorithm has a total computational time and space of $O(\sqrt{|G|})$, which means that in a group of order $2^{80}$ an attacker would approximately need $2^{40}$ operations, easily obtainable with modern hardware.

Direct consequence of this is that groups need to have a cardinality of at least $|G| \geq 2^{160}$, to make Shank's method have a complexity of $2^{80}$ and therefore be infeasible. In case of prime groups $G = \mathbb{Z}^*_p$, $p$ must have a length of at least 160 bit; using prime groups is advised since otherwise Shank's method would have an even smaller complexity. 

\subsection{Pollard's Rho method}
Pollard's Rho method has the same computational time as Shank's algorithm, yet constant space requirements. It is based on the \textbf{birthday paradox}, concept related to the higher likelihood of collisions found between random attack attempts and a fixed degree of permutations.

The basic idea consists in pseudo-randomly generate ($\rho$ represents randomness) group elements of the form $\alpha^i \cdot \beta^j$, keeping track of values $i$ and $j$, to then continue until obtaining a collision:
$$\alpha^{i_1} \cdot \beta^{j_1} = \alpha^{i_2} \cdot \beta^{j_2}$$
Substituting $\beta = \alpha^x$ and comparing the exponents on both sides of the equation, the collision leads to the following formula:
$$i_1 + xj_1 \equiv i_2 + jx_2 \mod |G| \qquad \rightarrow \qquad x \equiv \frac{i_2 - i_1}{j_1 - j_2} \mod |G|$$ 
The second part allows to find the discrete logarithm. Furthermore, additional methods to speed up computations such as the Extended Euclidean Algorithm and Floyd's cycle-finding algorithm.

Its computational time is approximately $O(\sqrt{n})$, while used together with Silver-Pohlig-Hellman it can achieve $O(\sqrt{p})$ where $p$ is the largest prime factor of $n$.

The practical importance of Rho's method is it being the best-known algorithm in \textit{elliptic curve group}s, making 160 bit operands very popular within this cryptography; however, it is not the most powerful attack for discrete logarithm and it does not work well in a distributed environment.


